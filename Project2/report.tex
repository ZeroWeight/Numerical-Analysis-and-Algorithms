\documentclass[onecolumn,compsoc]{IEEEtran}
\usepackage{algorithm}
\usepackage{algorithmicx}
\usepackage{algpseudocode}
\usepackage{amsfonts}
\usepackage{amsmath}
\usepackage{ctex}
\usepackage{float}
\usepackage{graphicx}
\usepackage{url}
\renewcommand{\IEEEQED}{\IEEEQEDclosed}
\renewcommand{\d}{\mathrm{d}}
\renewcommand{\|}{\Bigg |}
\newcommand{\e}{\mathrm{e}}
\title{\begin{large}数值分析第二次大作业\end{large}\\ e的数值计算}
\author{张蔚桐\ 2015011493\ 自55}
\begin {document}
\maketitle
\tableofcontents
\clearpage
\section{Taylor级数展开求e}
\subsection{原理分析}
考虑对函数$f(x) = \exp(x)$进行Maclaurin展开
\begin{equation}\label{1eq1}
\exp(x) = \sum_{i=0}^{\infty} \frac{\d^if(x)}{i!\d x^i}\|_{x = 0}x^i =  \sum_{i=0}^{\infty}\frac{x^i}{i!}
\end{equation}
将$x = 1$带入(\ref{1eq1})式得到
\begin{equation}\label{1eq2}
\e = \exp(1) = \sum_{i=0}^{\infty}\frac{1}{i!}
\end{equation}
显然级数以阶乘的速度下降,因此可以通过级数求和来迅速计算$\e$,截取(\ref{1eq2})前$n$项得到
\begin{equation}\label{1eq3}
\e \approx \sum_{i=0}^{n}\frac{1}{i!}
\end{equation}
\subsection{数值计算方法}
(\ref{1eq3})式虽然在理论上具有可计算性,然而对于$n!$的求取会对计算带来巨大的困难,实际上,$20!$已经超出了 C++标准
定义的long long长整型的表示范围,$\frac{1}{i!}$则会更早的由于有效数字的问题带来精度丢失。因此我们需要对(\ref{1eq3})
式进行计算上的处理。

展开(\ref{1eq3})式,得到
\begin{equation}\label{1eq4}
\e \approx 1 + 1 + \frac{1}{2!} + \frac{1}{3!} + \cdots + \frac{1}{n!}
\end{equation}
类似秦九韶算法,对计算次序进行优化得到(\ref{1eq5})式
\begin{equation}\label{1eq5}
\e\approx 2 + \frac{1}{2} (1 + \frac{1}{3}(1 + \cdots (\frac{1}{n-1}(1 + \frac{1}{n}))))
\end{equation}
采用逐次递推阶乘的原算法需要消耗$\mathcal{O}(n)$次加法,$\mathcal{O}(n)$次除法,而采用了次序优化之后的算法仍需要
$\mathcal{O}(n)$次加法,$\mathcal{O}(n)$次除法,且常数没有发生变化,然而此时的计算已经从根本上减少了大数加小数的误差,
而这种误差在顺行计算原公式时是很明显的。
\subsection{方法误差分析}
引入Maclaurin展式的Lagrange余项对方法误差进行分析
\begin{equation}\label{1eq6}
\exp(x) = \sum_{i=0}^n \frac{\d^if(x)}{i!\d x^i}\|_{x = 0}x^i + \frac{\d^{n+1}f(x)}{(n+1)!\d x^{n+1}}\|_{x = \xi}\xi^{n+1}, \xi \in(0,x)
\end{equation}
带入相关参数到(\ref{1eq6})式中得到
\begin{equation}\label{1eq7}
\e=\exp(1)=\sum_{i=0}^n \frac{1}{i!} + \exp(\xi)\frac{\xi^{n+1}}{(n+1)!}, \xi\in(0,1)
\end{equation}
因此得到误差项为(\ref{1eq8})式
\begin{equation}\label{1eq8}
R(n,\xi) = \exp(\xi)\frac{\xi^{n+1}}{(n+1)!}
\end{equation}
显然$R(n,\xi)$对于$\xi$而言单调递增,因此得到误差项上界为(\ref{1eq9})式
\begin{equation}\label{1eq9}
R(n) \le \sup_{\xi\in(0,1)}\bigg(\exp(\xi)\frac{\xi^{n+1}}{(n+1)!}\bigg) = \frac{\e}{(n+1)!}
\end{equation}
从(\ref{1eq9})式可以看出,误差随着计算次数$n$的增大而明显增大,从附录中我们知道double(long double)类型可以得到的15位有效数字,
又知道$2 < \e < 3$,因此我们只需要保证$R(n) \le 10^{-16}$既可以保证所有的有效数字均被利用,下面求取$n$的范围:

\begin{IEEEproof}[$n$范围的求取]

$$\frac{\e}{(n+1)!} \le 10^{-16} \Rightarrow 16\ln(10) \le \ln((n+1)!) - 1$$ 
应用Stirling公式近似,方便起见记$m = n+ 1$得到约束条件变为
$$16\ln(10) + 1 \le \ln(\sqrt{2\pi m}(\frac{m}{\e})^m) = \ln(\sqrt{2\pi m}) + m\ln(\frac{m}{\e}) = \ln(\sqrt{2\pi m}) + m\ln(m) - m$$
整理约束条件为$$16\ln(10) + 1 -\frac{\ln(2\pi)}{2}\le (m+\frac{1}{2})\ln(m) - m$$
求方程数值解可以得到$m \ge  15 \Rightarrow n \ge 14$
\end{IEEEproof}	
从上述证明可以看出当$n \ge 14$时,方法误差就小于了存储误差下界,因此可以看出误差的收敛性是很好的。实际计算中$n = 100$远远超出了这个精度。
\subsection{存储误差分析}
这里分析(\ref{1eq5})式带来的存储误差问题,我们依赖的数据结构为C++ 中的double(long double),他的存储有效位数请见附录。首先我们要给出算法的整体流程如算法\ref{A1}:
\begin{algorithm}[h]\caption{Taylor级数计算e}\label{A1}
\begin{algorithmic}
\Require 迭代次数$n$
\Ensure e的数值计算值eTaylor
\State 初始化 eTaylor = $\frac{1}{n+1}$
\For{$i$ 从 $n$到1}
\State 递增eTaylor
\State eTaylor = eTaylor $/i$
\EndFor
\State 递增eTaylor
\State \Return eTaylor
\end{algorithmic}\end{algorithm}
我们对算式中的每一步进行分析,首先初始化eTaylor过程中,eTaylor的有效数字仍保持15位,在第一个循环开始之后,递增eTaylor过程中出现第一次精度损失,将之前的从$\frac{1}{n+1}$开始的15位有效数字和1相加,必然得到从1开始的十五位有效数字,亦即精确到$10^{-14}$。另外,显然eTaylor此时在1 和2 中间,下一步除法时,有效数字仍为15位。

第二个循环开始之后,首先递增过程依然将精度损失到$10^{-14}$,之后每次循环均保持这这个运算有效数字,直到循环结束,此时eTaylor的值为e-1,下一步递增工作仍然保持着上述精度,因此运算精度可以保留到$10^{-14}$,即小数点后14位上。

上述分析的是单步存储误差,然而由于程序不断迭代,存储误差会被不断增大,因此需要估计迭代传递的存储误差,这里我们记$\Delta =5\times10^{-15}$估计为每步的有效数字引起的误差(上面已经证明运算精度可以达到$10^{-14}$)

主要的迭代过程是一个反向迭代,可以表示为(\ref{A1eq})式
\begin{equation}\label{A1eq}y_{n-1} = \frac{y_n + 1}{n}\end{equation}
取存储误差和传递误差分析得到误差传递为(\ref{A2eq})所示
\begin{equation}\label{A2eq}\delta_{y_{n-1}} = \frac{\delta_{y_{n}}}{n}+ \Delta\end{equation}
处理这个式子立刻得到可递推的(\ref{A3eq})式
\begin{equation}\label{A3eq}\frac{\delta_{y_{n-1}}}{(n-1)!} = \frac{\delta_{y_{n}}}{n!}
+ \frac{\Delta}{(n-1)!}\end{equation}
反向计算等差数列可以得到确定的传递误差精度为(\ref{A4eq})式
\begin{equation}\label{A4eq}\delta_{y_{n-1}} = \Delta\sum_{i=0}^n\frac{1}{i!}\approx{\Delta\e}\end{equation}
进一步可以说明传递误差精度是受限的,因此上面分析的单步误差精度是有意义的。
\subsection{计算结论分析和复杂度分析}
综合前两节的叙述,采用$n = 100$的设定之后,使用Taylor公式对e的计算可以达到小数点后15位,实际编程操作后确认了这个结论,我们得到的eTaylor和实际的e分别为(\ref{1eq10})式所示。程序耗时$1.2\mu s$,测试CPU主频2GHz
\begin{equation}\label{1eq10}\begin{aligned}
\mathrm{eTaylor} &= 2.71828\ 18284\ 59045 \\ 
\e &\approx 2.71828\ 18284\ 59045\ 23536\ 02874 
\end{aligned}\end{equation}
程序属于线性复杂度(按照加法和乘法次数计算),可推广性很好。且收敛速度为$\mathcal{O}(n!)$,不需要很大的$n$就可以让计算收敛。同时,Taylor方法的另外一个优点是,计算精度随着$n$的增大而单调递增,由于较大的$n$必然包括较小的$n$的计算项,提高$n$的值只可能是的后面的项受到计算精度的影响而消失,不影响前面的项。
\section{数值求解微分方程得到e的数值解}
\subsection{原理分析}
考虑含边界条件的微分方程(\ref{2eq1})式的解析解:
\begin{equation}\label{2eq1}
\frac{\d y}{\d x} = y;\  y(0) = 1
\end{equation}
自然得到方程的解析解为$y = \exp(x)$,计算$y(1)$即可得到e的值,因此只需要设计计算(\ref{2eq1})式的方法
\subsection{数值计算方法}
采用四阶Runge-Kutta方法对微分方程(\ref{2eq1})式进行求取,其常用的基本迭代方式为(\ref{2eq2})式所示,其中$h$为分割的小区间长度。
\begin{equation}\label{2eq2}\begin{cases}\begin{aligned}
y_{n+1} &= y_n + \frac{h}{6}(K_1+2K_2+2K_3+K_4)\\
K_1 &= y_n\\
K_2 &= y_n + \frac{h}{2}K_1 \\
K_3 &= y_n + \frac{h}{2}K_2 \\
K_4 &= y_n + hK_3
\end{aligned}\end{cases}\end{equation}
将区间从0到1分割成n段,按照递推进行计算即可得到e的值
\subsection{方法误差分析}
参考数值分析教材可以知道,四阶Runge-Kutta方法的截断误差为$\mathcal{O}(h^5)$,因此对整个区间上的截断误差为$\mathcal{O}(h^4)$,当$n >10^4$即$h < 10^{-4}$时,可以保留到16位有效数字,此时方法误差上限小于存储误差下限
\subsection{存储误差分析}
由递推式(\ref{2eq2})式知道,$y,K$这些参数受到加法的影响,其有效数字被确定在了15位,可以精确到小数点后14位,因此这个算法的计算精度应为小数点后14位。当然,如果$h$过小,$h$的作用就会被明显的削弱,由于$1 \le y \le \e$,而可以看出在$K_4$项上出现了$h^3$项,转到$y$的递推式上就出现了$h^4$项,因此我们必须保证$h^4$在有效数字范围内,同时从上节方法误差分析中我们可以看出$h < 10^{-4}$时才能保证方法误差小于存储误差,因此作为一种折中,实验操作中$n = 10^4,h < 10^{-4}$

上述分析的是单步存储误差,然而由于程序不断迭代,存储误差会被不断增大,因此需要估计迭代传递的存储误差,这里我们记$\Delta =5\times10^{-15}$估计为每步的有效数字引起的误差(上面已经证明运算精度可以达到$10^{-14}$)

我们按照(\ref{2eq2})式定义的方式来计算由于迭代造成的存储误差。首先误差分析得到(\ref{A5eq})式
\begin{equation}\label{A5eq}\begin{cases}\begin{aligned}
\delta_{y_{n+1}} &= \delta_{y_n} + \frac{h}{6}(\delta_{K_1}+2\delta_{K_2}+2\delta_{K_3}+\delta_{K_4}) + \Delta\\
\delta_{K_1} &= \delta_{y_n} \ \ \text{(由于没有计算过程,不含$\Delta$项)}\\
\delta_{K_2} &= \delta_{y_n} + \frac{h}{2}\delta_{K_1} + \Delta\\
\delta_{K_3} &= \delta_{y_n} + \frac{h}{2}\delta_{K_2} + \Delta\\
\delta_{K_4} &= \delta_{y_n} + h\delta_{K_3} + \Delta
\end{aligned}\end{cases}\end{equation}
将(\ref{A5eq})式带入分析得到(\ref{A6eq})式
\begin{equation}\label{A6eq}\begin{cases}\begin{aligned}
\delta_{K_2} &= (1+\frac{h}{2})\delta_{y_n} + \Delta\\
\delta_{K_3} &= (1+\frac{h}{2}+\frac{h^2}{4})\delta_{y_n} + (1+\frac{h}{2})\Delta\\
\delta_{K_4} &= (1+h+\frac{h^2}{2}+\frac{h^3}{4})\delta_{y_n} + (1+h+\frac{h^2}{2})\Delta\\

\delta_{y_{n+1}} &=\delta_{y_n} + \frac{h}{6}((5+3h+h^2+\frac{h^3}{4})\delta_{y_n} + (5+2h+\frac{h^2}{2})\Delta) + \Delta \le (1+h)(\delta_{y_n}+\Delta)
\end{aligned}\end{cases}\end{equation}
式(\ref{A6eq})中最后一步进行了一次近似,方便之后的求取。配系数得到(\ref{A7eq})式
\begin{equation}\label{A7eq}\begin{aligned}\delta_{y_{n+1}} &= (1+h)\delta_{y_n}+(1+h)\frac{(1+h)-1}{h}\Delta\\
\delta_{y_{n+1}} + \frac{1+h}{h}\Delta &= (1+h)(\delta_{y_{n}} + \frac{1+h}{h}\Delta)
\end{aligned}\end{equation}
利用等比数列相关知识立刻得到最终的误差分析式为(\ref{A8eq})式。
\begin{equation}\label{A8eq}
\delta_{y_{n}} = (1+h)^n(\frac{1+2h}{h}\Delta) - \frac{1+h}{h}\Delta\approx\frac{\e+2h\e-1-h}{h}\Delta\end{equation}

我们可以明确的发现这个式(\label{A8eq})是发散的。这说明h的值不能取得太小。当然式(\label{A8eq})只是一个近似式。通过上文关于主项的分析我们大致可以判断出$h$取值为多少时可以保证上式的相对稳定。综合所有分析可以取$h=10^{-4}$
\subsection{计算结论分析和复杂度分析}
综合前两节的叙述,采用$n =  10^4$的设定之后,使用微分方程对e的计算可以达到小数点后14位,实际编程操作后确认了这个结论,我们得到的eDEQN和实际的e分别为(\ref{2eq10})式所示。程序耗时$243.3\mu s$,测试CPU主频2GHz
\begin{equation}\label{2eq10}\begin{aligned}
\mathrm{eDEQN} &= 2.71828\ 18284\ 59040 \\ 
\e &\approx 2.71828\ 18284\ 59045\ 23536\ 02874 
\end{aligned}\end{equation}
程序属于线性复杂度(按照加法和乘法次数计算),可推广性较好,相比于Taylor方法的$\mathcal{O}(n!)$的收敛速度,这种方式的收敛速度是$\mathcal{O}(n^4)$,因此得到较好解的速度比上种方法慢得多。同时,相比于Taylor展开方法,一味提高$n$的值可能使计算过程中$h$项由于舍入误差消失,因此这种方法不如上面Taylor展开方法。
\section{数值积分求解e的数值解}
\subsection{原理分析}
考虑积分方程(\ref{3eq1})式的根为e,因此可以考虑通过求解此积分方程来计算e的数值解。
\begin{equation}\label{3eq1}
f(x) = \int_1^x\frac{1}{t}\d t = 1
\end{equation}
\subsection{数值计算方法}
首先我们对$f(x)$以$h$为间隔进行采样得到一组$f(x_k)$,显然他们之间的递推关系为(\ref{3eq2})式
\begin{equation}\label{3eq2}\begin{cases}
f(1) = 0\\
f(x_k) = f(x_{k-1}) + \int_{x_{k-1}}^{x_k}\frac{1}{t}\d t
\end{cases}\end{equation}
利用数值积分方法求取每一个小区间上的积分值,为了提高精度,我们可以将$h$取的充分小,并利用梯形公式求解。这样得到了一组$f(x_k)$。

通过寻找$f(x_k)-1$的过零点位置$x_z$进行线性插值(弦截法求根)可以得到零点$x^*$的位置为(\ref{3eq3})式
\begin{equation}\label{3eq3}\begin{cases}
f(x_z) > 1\\
f(x_{z-1}) < 1\\
x^* = x_{z-1} + \frac{1-f(x_{z-1})}{f(x_z)-f(x_{z-1})}
\end{cases}\end{equation}
积分公式采用梯形公式,因为这不需要增加内部节点的数量,相对速度也比较快,实际上,由于$h$已经取的很小,Cotos,Simpson,梯形三种积分公式的差别不大。
\subsection{方法误差分析}
方法误差包括两个方面,一方面是积分公式带来的误差,一方面是插值带来的误差。积分公式如果采用梯形公式,则在分割节点的控制之下
积分公式表现为复合梯形公式,其余项可以表征为(\ref{3eq4})式,其中积分函数$f(t) = 1 / t$
\begin{equation}\label{3eq4}
|R_n(f)| = -\frac{\e-1}{12}h^2f''(\xi)\|_{\xi\in(0,\e+h)} < \frac{\e-1}{6}h^2
\end{equation}
如果采用复合Simpson公式,则余项可以表征为(\ref{3eq5})式,其中积分函数$f(t) = 1 / t$
\begin{equation}\label{3eq5}
|R_n(f)| = -\frac{\e-1}{180}\frac{h}{2}^4f^{(4)}(\xi)\|_{\xi\in(0,\e+h)} < \frac{\e-1}{120}h^4
\end{equation}
可以看到复合Simpson公式带来更小的误差

另一部分的误差来源于插值求根公式,线性插值误差表征为(\ref{3eq6})式,其中插值原函数$f(x) = \ln(x)$
\begin{equation}\label{3eq6}
R_n(x) = |\frac{f''(\xi)}{2}(x-x_1)(x-x_1)|\|_{\xi \in (\e-h,\e+h)} \le \frac{h^2}{8\e^2}
\end{equation}

综合两方面的误差,我们可以认为误差在$\mathcal O (h^2)$量级上,当$n = 1/h = 10^8$时,方法误差上限小于舍入误差下限。
\subsection{存储误差分析}
首先写出我们的算法过程如算法\ref{A2}所示
\begin{algorithm}[h]\caption{数值积分计算e}\label{A2}
\begin{algorithmic}
\Require 分割间隔$N$
\Ensure e的数值计算值eInt
\State 初始化$iter = 1, h = \frac{1}{N}, p = 0, n = 0$
\State 为$p$加下一个积分区间段的值
\While {$p$ < 1}
\State $n$ = $p$;
\State $iter$ += $h$;
\State 为$p$加下一个积分区间段的值
\EndWhile
\State 计算eInt$ = iter + h\frac{1 - n}{p - n}$
\State \Return eInt
\end{algorithmic}\end{algorithm}
我们可以看出,在计算n和p的过程中,收到加法的限制,$p$最终可以保证精确到小数点后14位,
$n$最终可以保证精确到小数点后15位,精度的损失出现在线性插值过程中。$\frac{1 - n}{p - n}$
将出现位数损失,根据程序计算流程可以得到$\ln(\e-h) < n < 1 < p < \ln(\e+h)$。因此得到
\begin{equation}\label{3eq7}\begin{cases}
-\frac{h}{\e} < \ln(\e-h) -1 < n -1 < 0 \Rightarrow 1-n < \frac{h}{\e} \\
0 < p-n < 2\frac{h}{\e}
\end{cases}\end{equation}
$1-n$和$p-n$均只能在(\ref{3eq7})式确定的值,但只能保留到小数点后14位。
这里做一个近似的估计,我们设$N = \frac{1}{h} = 10^m$,则可以看出有效数字缩减为$14 - m$位

但是受到前面iter项的限制,存储精度可以保留到小数点后14位。

下面讨论积分公式中的存储误差限制,积分原函数为$f(t) = 1/t$,要求$t$变化时,变化的主项必须保证,
也就是说$f(t+h) - f(t)$的数值必须非零,否则会产生明显的存储误差。利用Taylor公式展开得到
$f(t+h) - f(t) = -\frac{h}{t^2}$,因此我们需要要求$\frac{h}{t}>10^{-14}$,于是得到
$h > 10^{-13}$

下面分析迭代误差,由于程序计算过程中对于积分累加过程中将误差直接进行了累加,因此从过程中得到的累加误差高达$n\Delta$(记$\Delta =5\times10^{-15}$估计为每步的有效数字引起的误差(上面已经证明运算精度可以达到$10^{-14}$))。因此积分方法的误差发散的也最严重。结合上面的方法误差可以达到$h^2$的分析,下面求解导致误差发散不会淹没方法的最小$h$:由于误差传递从$10^{-15}$量级上损失$h$个有效数字,而方法误差达到$h^2$量级,因此得到两个方法的交汇在$2\|\lg h\| = 15-\|\lg h\|$上,得到$h=10^{-5}$,对应方法误差和存储误差均在$10^{-10}$级别上,因此预计可以达到$10^{-9}$级别的精确度。和前两个相比相差很多。
\subsection{计算结论分析和复杂度分析}
由于涉及的分割区间大小$h$过小,因此在计算过程中根据系统实现的不同也会出现存储误差,经过测试,当
$n = \frac{1}{h} = 10^7$时出现明显的存储误差,主要的存储误差出现在了对$h = 1/n$的存储,大数和
小数的相加等问题,另外,这个算法的计算复杂度是$\mathcal O(N) = \mathcal O(\frac{1}{h})$的,当
$h$选取的过小时,计算时间出现明显的上升。实际测试中选取了$N = 10^5$,可以精确到小数点后9位。这一点和之前理论分析得到的估计值一致。

实际编程操作后确认了这个结论,我们得到的eInt和实际的e分别为(\ref{3eq10})式所示。可以精确到小数点后10位
程序耗时$13\mathrm{ms}$,测试CPU主频2GHz
\begin{equation}\label{3eq10}\begin{aligned}
\mathrm{eDEQN} &= 2.71828\ 18284\ 60041 \\ 
\e &\approx 2.71828\ 18284\ 59045\ 23536\ 02874 
\end{aligned}\end{equation}
程序属于线性复杂度(按照加法和乘法次数计算),相比于Taylor方法的$\mathcal{O}(n!)$的收敛速度,
这种方式的收敛速度是$\mathcal{O}(n^2)$,因此得到较好解的速度比上种方法慢得多。同时,相比于Taylor展开方法,
一味提高$n$的值可能使计算过程中$h$项由于舍入误差消失,因此这种方法不如上面两种方法。
\section{小结}
从上述讨论中我们可以看出,Taylor展开方法相对于后面两种方法有着更好的收敛速度,更低的计算代价和更小的计算误差,
因此在实际计算过程中更推荐使用Taylor展开方法计算e的数值
\appendix[C++ 标准中double的存储方式和有效值]
C++ 中的基本数据类型double(又称long double)是符合IEEE 754浮点数运算标准下的64位浮点数基本数据结构。
按照IEEE 754的要求,其包括1位二进制符号位,11位二进制指数位和52位二进制尾数位,其表示方法是二进制科学
计数法。因此可以认为double的二进制有效数字为52位,转10进制得到$\log_{10}52 \approx 15.6$。因此我们可以
认为一个double(long double)可以提供15位有效数字,实际的测试中可以验证这个有效数字位数是正确的。
\end{document}