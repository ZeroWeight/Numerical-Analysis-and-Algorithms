\documentclass[UTF8,a4paper]{paper}\usepackage[utf8]{inputenc}\usepackage{algorithm}
\usepackage{algorithmicx}\usepackage{algpseudocode}\usepackage{amsfonts}\usepackage{amsmath}
\usepackage{ctex}\usepackage{diagbox}\usepackage{float}\usepackage{graphicx}
\usepackage{listings}\usepackage{multicol}\usepackage{multirow}\usepackage{pdfpages}
\usepackage{rotating}\usepackage{url}\usepackage{wrapfig}
\title{\begin{large}数值分析第一次大作业\end{large}\\ 图像扭曲变形
\\ \begin{large}程序技术报告\end{large}}
\author{张蔚桐\ 2015011493\ 自55}
\begin {document}
\maketitle
\section{开发环境和运行平台}
程序采用C\#基于Visual Studio 2017开发,需要在安装了.NET framework 4.0 以上版本的运行。
在Windows 7及以下系统内可能需要重新安装.NET framework 4.0,在Windows 8及以上版本中可以直接运行。
如果不能直接运行请检查.NET framework 版本。工程文件可以采用Visual Studio 2015 及以上版本直接打开。
\section{程序使用说明}
使用前段设计思路对程序进行了优化,使得程序更加用户友好。

通过UI控件调整每一个变换的参数以及插值方法,对于旋转变换和水波纹变换,在图像上单击可以确定变换中心,
拖动鼠标(不需要长按左键)可以选择变换半径,程序会自动显示变换的区间。再次点击确定变换的半径,程序
开始运行。

对B样条插值,点击B-spline开始工作,程序会渲染出所有的控制点。在控制点附近单击鼠标左键,程序会自动捕获
最近的控制点并将其渲染成红色。拖动这个控制点,松开鼠标后程序开始执行。

在任何涉及到鼠标的操作的过程中,单击鼠标右键可以取消当前的所有操作。

完成之后,点击 'keep changes' 按钮可以将当前图像暂存,可以在已经修改的图像的基础上继续进行修改,而点击
'discard changes' 按钮可以将图片回复到最原始的形态,最原始的图片始终显示在左上角。

如果由于用户的不当操作导致图像的历史记录出现了混乱(如,没有回退到原始图像等),重新选择加载图片可以解决几乎
所有问题。点击 'save image' 可以保存当前工作界面上的图片。

点击close 按钮可以关闭窗口,点击minimize按钮可以最小化窗口,在非控件上直接拖动可以改变窗口的位置。实现了扁平化的
设计思想。
\section{程序执行速度说明}
由于C\#的高度封装性,相比C++的执行速度比较慢,尤其是在遍历每一个像素的时候,由于C\#访问的并不是raw data,使得
处理过程可能显得比较长,可以参考右下角进度条了解程序的执行流程。
\section{代码情况}
主要的内容均在Form1.cs文件中,使用Region分割成了若干块,其中kernel是主要部分,B样条插值的实现过程
主要在mouse up函数中。
\section{压缩包内容}
测试图片(2张)在data/originIMG下,之前生成的图片在data/genIMG下(已经写到报告中),ImagePro/下是工程文件
\end{document}